\documentclass[10pt,a4paper]{article}

\usepackage{resume}

\begin{document}
\thispagestyle{empty}

\begin{minipage}[b]{0.72\linewidth}
  \textcolor{AccentDeep}{\rule{\linewidth}{1.2pt}}\\[4pt]
  \furiname{オウ}{イシン}{王}{一臻}\\[4pt]
  {\normalsize WANG\;Yizhen}

  \vspace{6pt}
  \rolebadge{修士 | 東京大学 長谷川研究室 | 機械工学}\\[4pt]
  {\footnotesize
    \contactitem{\faEnvelope[regular]}{\href{mailto:wang1zhen97@gmail.com}{wang1zhen97@gmail.com}}\hspace{1em}
    \contactitem{\faPhone}{080-8008-2338}\hspace{1em}
    \contactitem{\faMapMarker*}{日本・東京}\\[2pt]
    \contactitem{\faGlobe}{\href{https://blog.wang1zhen.com}{blog.wang1zhen.com}}\hspace{1em}
    \contactitem{\faGithub}{\href{https://github.com/wang1zhen}{github.com/wang1zhen}}\hspace{1em}
    \contactitem{\faLinkedin}{\href{https://www.linkedin.com/in/\%E4\%B8\%80\%E8\%87\%BB-yizhen-\%E3\%82\%A4\%E3\%82\%B7\%E3\%83\%B3-\%E7\%8E\%8B-wang-\%E3\%82\%AA\%E3\%82\%A6-3a7b68110/}{LinkedIn}}
  }
\end{minipage}
\hfill
\begin{minipage}[b]{0.22\linewidth}
  \raggedleft
  \raisebox{5pt}{\includegraphics[width=3.0cm]{1.jpg}}
\end{minipage}

\vspace{6pt}
\cvsection{学歴}

\eduentry
  {東京大学}
  {修士課程(機械工学専攻、長谷川研究室)}
  {日本・東京}
  {2023.10 -- 2025.9}
  {\item 円管乱流実験における抵抗低減のための脈動波形の自動最適化
   \item \textbf{修士論文が日本自動車技術会(JSAE)の Award of Society of Automotive Engineers of Japan を受賞}}

\eduentry
  {メルボルン大学}
  {修士課程(機械工学専攻、中退)}
  {オーストラリア・メルボルン}
  {2019.1 -- 2019.12}
  {\item 粗面高レイノルズ数乱流境界層におけるエネルギー運動の解明}

\eduentry
  {上海交通大学}
  {学士課程(エネルギー・動力工学専攻、音楽学副専攻)}
  {中国・上海}
  {2014.9 -- 2018.9}
  {\item DMEおよびエタノールによるメタン火炎の希釈効果
   \item 水素冷却発電機の固定子基礎の動特性}

\eduentry
  {上海市実験学校}
  {高等学校卒業}
  {中国・上海}
  {2011.9 -- 2014.8}
  {}

\vspace{6pt}
\setlength{\columnsep}{2\dimexpr\columnsep\relax}
\begin{multicols}{2}
\cvsection{スキル}
\begin{itemize}[itemsep=5pt,topsep=2pt]
  \item \textbf{機械工学}\\
        ツール: SolidWorks, OpenFOAM, ANSYS\\
        履修科目: 数値熱流体工学、伝熱工学特論、\\
        流体工学特論、数値解析
  \item \textbf{電気工学}\\
        履修科目: 回路理論、PLC制御の可視化、\\
        計測原理と計測技術、\\
        計装概論とシステム・制御モデリング
  \item \textbf{機械学習・最適化}\\
        手法・履修科目: ベイズ最適化、\\
        脳型情報処理機械論、神経知能
  \item \textbf{その他}\\
        プログラミング: Python, MATLAB, C/C++,\\
        Emacs Lisp\\
        ツール: Microsoft Office, \LaTeX\\
        システム: GNU/Linux, openWRT, BSD
\end{itemize}

\columnbreak
\cvsection{語学力}
\begin{itemize}
  \item 中国語:母国語
  \item 英語:ネーティブレベル
  \item 日本語:JLPT N2 (CEFR B2, ビジネスレベル)
\end{itemize}

\cvsection{趣味}
\begin{itemize}[itemsep=4pt,topsep=2pt]
  \item クラシック音楽:アマチュアビオラ奏者
  \item パソコンDIY:サーバー管理者
  \item 推理小説とSF作品
\end{itemize}
\end{multicols}

\end{document}
