%%%%%%%%%%%%%%%%%
% This is an sample CV template created using altacv.cls
% (v1.7.2, 28 August 2024) written by LianTze Lim (liantze@gmail.com). Compiles with pdfLaTeX, XeLaTeX and LuaLaTeX.
%
%% It may be distributed and/or modified under the
%% conditions of the LaTeX Project Public License, either version 1.3
%% of this license or (at your option) any later version.
%% The latest version of this license is in
%%    http://www.latex-project.org/lppl.txt
%% and version 1.3 or later is part of all distributions of LaTeX
%% version 2003/12/01 or later.
%%%%%%%%%%%%%%%%

%% Use the "normalphoto" option if you want a normal photo instead of cropped to a circle
% \documentclass[10pt,a4paper,normalphoto]{altacv}

\documentclass[10pt,a4paper,ragged2e,withhyper]{altacv}
%% AltaCV uses the fontawesome5 and simpleicons packages.
%% See http://texdoc.net/pkg/fontawesome5 and http://texdoc.net/pkg/simpleicons for full list of symbols.

% Change the page layout if you need to
\geometry{left=1.25cm,right=1.25cm,top=1.5cm,bottom=1.5cm,columnsep=1.2cm}

% The paracol package lets you typeset columns of text in parallel
\usepackage{paracol}

% Change the font if you want to, depending on whether
% you're using pdflatex or xelatex/lualatex
% WHEN COMPILING WITH XELATEX PLEASE USE
% xelatex -shell-escape -output-driver="xdvipdfmx -z 0" main_en.tex
\iftutex
  % If using xelatex or lualatex: use system fonts
  \setmainfont{DejaVu Serif}
  \setsansfont{DejaVu Sans}
  \renewcommand{\familydefault}{\sfdefault}
\else
  % If using pdflatex:
  \usepackage[rm]{roboto}
  \usepackage[defaultsans]{lato}
  % \usepackage{sourcesanspro}
  \renewcommand{\familydefault}{\sfdefault}
\fi

% Change the colours if you want to
\definecolor{SlateGrey}{HTML}{2E2E2E}
\definecolor{LightGrey}{HTML}{666666}
\definecolor{DarkPastelRed}{HTML}{450808}
\definecolor{PastelRed}{HTML}{8F0D0D}
\definecolor{GoldenEarth}{HTML}{E7D192}
\definecolor{MutedNavy}{HTML}{233447}
\colorlet{name}{black}
\colorlet{tagline}{PastelRed}
\colorlet{heading}{DarkPastelRed}
\colorlet{headingrule}{GoldenEarth}
\colorlet{subheading}{PastelRed}
\colorlet{accent}{PastelRed}
\colorlet{emphasis}{MutedNavy}
\colorlet{body}{LightGrey}

% Change some fonts, if necessary
\renewcommand{\namefont}{\Huge\rmfamily\bfseries}
\renewcommand{\personalinfofont}{\footnotesize}
\renewcommand{\cvsectionfont}{\LARGE\rmfamily\bfseries}
\renewcommand{\cvsubsectionfont}{\large\bfseries}


% Change the bullets for itemize and rating marker
% for \cvskill if you want to
\renewcommand{\cvItemMarker}{{\small\textbullet}}
\renewcommand{\cvRatingMarker}{\faCircle}
% ...and the markers for the date/location for \cvevent
% \renewcommand{\cvDateMarker}{\faCalendar*[regular]}
% \renewcommand{\cvLocationMarker}{\faMapMarker*}


% If your CV/résumé is in a language other than English,
% then you probably want to change these so that when you
% copy-paste from the PDF or run pdftotext, the location
% and date marker icons for \cvevent will paste as correct
% translations. For example Spanish:
% \renewcommand{\locationname}{Ubicación}
% \renewcommand{\datename}{Fecha}


%% Use (and optionally edit if necessary) this .tex if you
%% want to use an author-year reference style like APA(6)
%% for your publication list
% \input{pubs-authoryear.tex}

%% Use (and optionally edit if necessary) this .tex if you
%% want an originally numerical reference style like IEEE
%% for your publication list
\input{pubs-num.tex}

%% sample.bib contains your publications
\addbibresource{sample.bib}
% \usepackage{academicons}\let\faOrcid\aiOrcid
\begin{document}
\name{WANG Yizhen}
\tagline{Master, Hasegawa Lab, The University of Tokyo}
%% You can add multiple photos on the left or right
\photoR{3.0cm}{1.jpg}
% \photoL{2.5cm}{Yacht_High,Suitcase_High}

\personalinfo{%
  % Not all of these are required!
  \email{wangyz@iis.u-tokyo.ac.jp}
  \phone{080-8008-2338}
  \location{Tokyo, Japan}\\
  \homepage{blog.wang1zhen.com}
  \github{wang1zhen}
  \printinfo{\faLinkedin}{wang1zhen}[https://www.linkedin.com/in/\%E4\%B8\%80\%E8\%87\%BB-yizhen-\%E3\%82\%A4\%E3\%82\%B7\%E3\%83\%B3-\%E7\%8E\%8B-wang-\%E3\%82\%AA\%E3\%82\%A6-3a7b68110/]
  %% You can add your own arbitrary detail with
  %% \printinfo{symbol}{detail}[optional hyperlink prefix]
  % \printinfo{\faPaw}{Hey ho!}[https://example.com/]

  %% Or you can declare your own field with
  %% \NewInfoFiled{fieldname}{symbol}[optional hyperlink prefix] and use it:
  % \NewInfoField{gitlab}{\faGitlab}[https://gitlab.com/]
  % \gitlab{your_id}
  %%
  %% For services and platforms like Mastodon where there isn't a
  %% straightforward relation between the user ID/nickname and the hyperlink,
  %% you can use \printinfo directly e.g.
  % \printinfo{\faMastodon}{@username@instace}[https://instance.url/@username]
  %% But if you absolutely want to create new dedicated info fields for
  %% such platforms, then use \NewInfoField* with a star:
  % \NewInfoField*{mastodon}{\faMastodon}
  %% then you can use \mastodon, with TWO arguments where the 2nd argument is
  %% the full hyperlink.
  % \mastodon{@username@instance}{https://instance.url/@username}
}

\makecvheader
%% Depending on your tastes, you may want to make fonts of itemize environments slightly smaller
% \AtBeginEnvironment{itemize}{\small}

%% Set the left/right column width ratio to 6:4.
\columnratio{0.6}

% Start a 2-column paracol. Both the left and right columns will automatically
% break across pages if things get too long.
\begin{paracol}{2}
\cvsection{Education}

\cvevent{High School}{Shanghai Experimental School}{2011.9 -- 2014.8}{Shanghai, China}

\divider

\cvevent{Bachelor's Degree}{Shanghai Jiao Tong University}{2014.9 -- 2018.9}{Shanghai, China}
\begin{itemize}
\item School of Mechanical Engineering, Major in Energy and Power Engineering
\item School of Humanities, Minor in Music
\end{itemize}
{\small\color{emphasis}{$\triangleright$ The dilution effect of DME and ethanol on methane flames\\$\triangleright$ Dynamic characteristics of a hydrogen-cooling generator stator base}}

\divider

\cvevent{Master's Degree (Withdrawn)}{The University of Melbourne}{2019.1 -- 2019.12}{Melbourne, Australia}
\begin{itemize}
\item Major in Mechanical Engineering
\end{itemize}
{\small\color{emphasis}{$\triangleright$ Energetic motions in rough-wall high-Reynolds-number turbulent boundary layers}}

\divider

\cvevent{Master's Degree}{The University of Tokyo}{2023.10 -- 2025.9}{Tokyo, Japan}
\begin{itemize}
\item Major in Mechanical Engineering
\end{itemize}
{\small\color{emphasis}{$\triangleright$ Automated pulsation-waveform optimization for drag reduction in pipe-flow turbulence experiments}}

\cvsection{Skills}

{\Large\color{accent}{Programming}}

{\large\color{emphasis}\cvtag{Python} \cvtag{MATLAB} \cvtag{C\&C++} \cvtag{E-LISP}}

\divider

{\Large\color{accent}{3D Modeling \& Simulation}}

{\large\color{emphasis}\cvtag{SolidWorks} \cvtag{OpenFOAM} \cvtag{ANSYS}}

\divider

{\Large\color{accent}{Office Software}}

{\large\color{emphasis}\cvtag{Microsoft Office} \cvtag{\LaTeX}}

\divider

{\Large\color{accent}{Others}}

{\large\color{emphasis}\cvtag{Linux} \cvtag{openWRT} \cvtag{BSD} \cvtag{NAS}}

% use ONLY \newpage if you want to force a page break for
% ONLY the current column
%\newpage

%% Switch to the right column. This will now automatically move to the second
%% page if the content is too long.
\switchcolumn

\cvsection{Life Philosophy}


\begin{quote}
\LARGE{Climb}
\end{quote}

\bigskip

% \cvsection{Strengths}

% % Don't overuse these \cvtag boxes — they're just eye-candies and not essential. If something doesn't fit on a single line, it probably works better as part of an itemized list (probably inlined itemized list), or just as a comma-separated list of strengths.

% % The `ragged2e` document class option might cause automatic linebreaks between \cvtag to fail.
% % Either remove the ragged2e option; or
% % add \LaTeXraggedright in the paragraph for these \cvtag
% {\LaTeXraggedright
% \cvtag{Persistent}
% \cvtag{Systems Thinking}
% \cvtag{Team Spirit}
% \par}

% \divider

% \smallskip

% %% ...Or manually add linebreaks yourself
% \cvtag{Python}
% \cvtag{C \& C++}
% \cvtag{\LaTeX}\\
% \cvtag{Emacs-Lisp}

% %% Yeah I didn't spend too much time making all the
% %% spacing consistent... sorry. Use \smallskip, \medskip,
% %% \bigskip, \vspace etc to make adjustments.
% \bigskip

\cvsection{Languages}

\cvskill{Chinese (Native)}{5}
\divider

\cvskill{English (GRE 320)}{4.5}
\divider

\cvskill{Japanese (JLPT N2)}{4}

\bigskip

\cvsection{Hobbies}

{\large\color{emphasis}
\begin{itemize}
\item \color{accent}{Classical Music}

\color{emphasis}{Amateur Viola Player}

\begin{center}
   \includegraphics[width=0.6\linewidth]{2.jpg}
\end{center}

\item \color{accent}{PC DIY}

\color{emphasis}{Server Administrator}

\begin{center}
   \includegraphics[width=0.6\linewidth]{3.jpg}
\end{center}

\item \color{accent}{Cats!}

\begin{center}
   \includegraphics[width=0.6\linewidth]{4.jpg}
\end{center}

\end{itemize}}

\end{paracol}

\end{document}
